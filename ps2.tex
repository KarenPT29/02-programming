% Options for packages loaded elsewhere
\PassOptionsToPackage{unicode}{hyperref}
\PassOptionsToPackage{hyphens}{url}
%
\documentclass[
  12pt,
]{article}
\usepackage{amsmath,amssymb}
\usepackage{iftex}
\ifPDFTeX
  \usepackage[T1]{fontenc}
  \usepackage[utf8]{inputenc}
  \usepackage{textcomp} % provide euro and other symbols
\else % if luatex or xetex
  \usepackage{unicode-math} % this also loads fontspec
  \defaultfontfeatures{Scale=MatchLowercase}
  \defaultfontfeatures[\rmfamily]{Ligatures=TeX,Scale=1}
\fi
\usepackage{lmodern}
\ifPDFTeX\else
  % xetex/luatex font selection
\fi
% Use upquote if available, for straight quotes in verbatim environments
\IfFileExists{upquote.sty}{\usepackage{upquote}}{}
\IfFileExists{microtype.sty}{% use microtype if available
  \usepackage[]{microtype}
  \UseMicrotypeSet[protrusion]{basicmath} % disable protrusion for tt fonts
}{}
\makeatletter
\@ifundefined{KOMAClassName}{% if non-KOMA class
  \IfFileExists{parskip.sty}{%
    \usepackage{parskip}
  }{% else
    \setlength{\parindent}{0pt}
    \setlength{\parskip}{6pt plus 2pt minus 1pt}}
}{% if KOMA class
  \KOMAoptions{parskip=half}}
\makeatother
\usepackage{xcolor}
\usepackage[margin=1in]{geometry}
\usepackage{color}
\usepackage{fancyvrb}
\newcommand{\VerbBar}{|}
\newcommand{\VERB}{\Verb[commandchars=\\\{\}]}
\DefineVerbatimEnvironment{Highlighting}{Verbatim}{commandchars=\\\{\}}
% Add ',fontsize=\small' for more characters per line
\usepackage{framed}
\definecolor{shadecolor}{RGB}{248,248,248}
\newenvironment{Shaded}{\begin{snugshade}}{\end{snugshade}}
\newcommand{\AlertTok}[1]{\textcolor[rgb]{0.94,0.16,0.16}{#1}}
\newcommand{\AnnotationTok}[1]{\textcolor[rgb]{0.56,0.35,0.01}{\textbf{\textit{#1}}}}
\newcommand{\AttributeTok}[1]{\textcolor[rgb]{0.13,0.29,0.53}{#1}}
\newcommand{\BaseNTok}[1]{\textcolor[rgb]{0.00,0.00,0.81}{#1}}
\newcommand{\BuiltInTok}[1]{#1}
\newcommand{\CharTok}[1]{\textcolor[rgb]{0.31,0.60,0.02}{#1}}
\newcommand{\CommentTok}[1]{\textcolor[rgb]{0.56,0.35,0.01}{\textit{#1}}}
\newcommand{\CommentVarTok}[1]{\textcolor[rgb]{0.56,0.35,0.01}{\textbf{\textit{#1}}}}
\newcommand{\ConstantTok}[1]{\textcolor[rgb]{0.56,0.35,0.01}{#1}}
\newcommand{\ControlFlowTok}[1]{\textcolor[rgb]{0.13,0.29,0.53}{\textbf{#1}}}
\newcommand{\DataTypeTok}[1]{\textcolor[rgb]{0.13,0.29,0.53}{#1}}
\newcommand{\DecValTok}[1]{\textcolor[rgb]{0.00,0.00,0.81}{#1}}
\newcommand{\DocumentationTok}[1]{\textcolor[rgb]{0.56,0.35,0.01}{\textbf{\textit{#1}}}}
\newcommand{\ErrorTok}[1]{\textcolor[rgb]{0.64,0.00,0.00}{\textbf{#1}}}
\newcommand{\ExtensionTok}[1]{#1}
\newcommand{\FloatTok}[1]{\textcolor[rgb]{0.00,0.00,0.81}{#1}}
\newcommand{\FunctionTok}[1]{\textcolor[rgb]{0.13,0.29,0.53}{\textbf{#1}}}
\newcommand{\ImportTok}[1]{#1}
\newcommand{\InformationTok}[1]{\textcolor[rgb]{0.56,0.35,0.01}{\textbf{\textit{#1}}}}
\newcommand{\KeywordTok}[1]{\textcolor[rgb]{0.13,0.29,0.53}{\textbf{#1}}}
\newcommand{\NormalTok}[1]{#1}
\newcommand{\OperatorTok}[1]{\textcolor[rgb]{0.81,0.36,0.00}{\textbf{#1}}}
\newcommand{\OtherTok}[1]{\textcolor[rgb]{0.56,0.35,0.01}{#1}}
\newcommand{\PreprocessorTok}[1]{\textcolor[rgb]{0.56,0.35,0.01}{\textit{#1}}}
\newcommand{\RegionMarkerTok}[1]{#1}
\newcommand{\SpecialCharTok}[1]{\textcolor[rgb]{0.81,0.36,0.00}{\textbf{#1}}}
\newcommand{\SpecialStringTok}[1]{\textcolor[rgb]{0.31,0.60,0.02}{#1}}
\newcommand{\StringTok}[1]{\textcolor[rgb]{0.31,0.60,0.02}{#1}}
\newcommand{\VariableTok}[1]{\textcolor[rgb]{0.00,0.00,0.00}{#1}}
\newcommand{\VerbatimStringTok}[1]{\textcolor[rgb]{0.31,0.60,0.02}{#1}}
\newcommand{\WarningTok}[1]{\textcolor[rgb]{0.56,0.35,0.01}{\textbf{\textit{#1}}}}
\usepackage{graphicx}
\makeatletter
\def\maxwidth{\ifdim\Gin@nat@width>\linewidth\linewidth\else\Gin@nat@width\fi}
\def\maxheight{\ifdim\Gin@nat@height>\textheight\textheight\else\Gin@nat@height\fi}
\makeatother
% Scale images if necessary, so that they will not overflow the page
% margins by default, and it is still possible to overwrite the defaults
% using explicit options in \includegraphics[width, height, ...]{}
\setkeys{Gin}{width=\maxwidth,height=\maxheight,keepaspectratio}
% Set default figure placement to htbp
\makeatletter
\def\fps@figure{htbp}
\makeatother
\setlength{\emergencystretch}{3em} % prevent overfull lines
\providecommand{\tightlist}{%
  \setlength{\itemsep}{0pt}\setlength{\parskip}{0pt}}
\setcounter{secnumdepth}{-\maxdimen} % remove section numbering
\usepackage{graphicx}
\usepackage{fancyhdr}
\usepackage{booktabs,xcolor}
\pagestyle{fancy}
\fancyhf{}
\rhead{Spring 2024}
\lhead{SIS-750-007: Data Analysis}
\fancypagestyle{plain}{\pagestyle{fancy}}
\setlength{\headheight}{14.49998pt}

\usepackage[shortlabels]{enumitem}
\setlist{nosep}
\ifLuaTeX
  \usepackage{selnolig}  % disable illegal ligatures
\fi
\IfFileExists{bookmark.sty}{\usepackage{bookmark}}{\usepackage{hyperref}}
\IfFileExists{xurl.sty}{\usepackage{xurl}}{} % add URL line breaks if available
\urlstyle{same}
\hypersetup{
  pdftitle={Problem Set 2},
  pdfauthor={Due 31 January},
  hidelinks,
  pdfcreator={LaTeX via pandoc}}

\title{Problem Set 2}
\author{Due 31 January}
\date{}

\begin{document}
\maketitle

Answer the questions below to the best of your ability. Write clearly,
and format your tables and visuals appropriately. You must use
\texttt{R\ Markdown} to compose and compile your work. For full credit,
\texttt{echo} all code chunks, and include your \texttt{setup} chunk.
Submit your work in hard copy at the beginning of class.

You need the
\href{https://dataverse.harvard.edu/dataset.xhtml?persistentId=doi:10.7910/DVN/TMWYHB}{Global
Greenspace Indicator Data} for this assignment. Review the
\texttt{README.txt} file for information about the data, variables, etc.

\begin{enumerate}
\def\labelenumi{\arabic{enumi}.}
\item
  Show me that you're all set on GitHub. Create a public repository
  named \texttt{PSet2}. Clone it, include all your project files
  (including your \texttt{.Rmd} and \texttt{.pdf} files) for the work
  below, and commit/push your work to your repository. Include the link
  to your repo as your answer to this question.
\item
  The script below doesn't work. Type the corrected code chunk into your
  problem set. Annotate any line you correct to note your fix
  (i.e.~\texttt{\#\ unbalanced\ parentheses}). \emph{There are more than
  five errors.}
\end{enumerate}

\begin{Shaded}
\begin{Highlighting}[]
  \FunctionTok{library}\NormalTok{(tidyverse)}

\CommentTok{\# open my data}
\NormalTok{  gspace }\OtherTok{=} \FunctionTok{read\_csv}\NormalTok{(greenspace\_data\_share.csv)}

\CommentTok{\# summarize average urban greenspace by region}
\NormalTok{  table }\OtherTok{=} 
\NormalTok{    gspace }\SpecialCharTok{|\textgreater{}}
    \FunctionTok{group\_by}\NormalTok{(Major\_Geo\_Region)}
    \FunctionTok{summarise}\NormalTok{(}
      \AttributeTok{obs =} \FunctionTok{n}\NormalTok{()}
      \AttributeTok{avg =} \FunctionTok{mean}\NormalTok{(annual\_avg2020),}
\NormalTok{      weighted }\AttributeTok{avg =} \FunctionTok{mean}\NormalTok{(annual\_weight\_avg\_2020)}
\NormalTok{    )}

\CommentTok{\# output as table}
  \FunctionTok{kable}\NormalTok{(gspace, }\AttributeTok{digits =} \DecValTok{1}\NormalTok{)}
\end{Highlighting}
\end{Shaded}

\begin{enumerate}
\def\labelenumi{\arabic{enumi}.}
\setcounter{enumi}{2}
\item
  How many urban areas does the greenspace data cover?
\item
  In a couple of sentences and with reference to a well-formatted
  tabulation, describe the greenspace classification scores for urban
  areas in 2021.
\item
  Report the number of urban areas that satisfy the conditions below.
  Either write your code inline or echo the code that generated the
  answer.

  \begin{enumerate}
  \def\labelenumii{\alph{enumii}.}
  \item
    Scored \texttt{High} or above for greenspace in 2015.
  \item
    Scored \texttt{Exceptionally\ Low} at any point in the years
    covered.
  \item
    Urban areas in arid climate that became greener (as measured by
    annual weighted average) from 2010 to 2020.
  \end{enumerate}
\item
  How many urban areas became less green (measured by annual average)
  from 2010 to 2021? Were these changes concentrated in a particular
  geographic or climate region? Explain (with evidence, of course).
\item
  Present a histogram showing the change in greenspace (annual average)
  from 2010 to 2021. Note that you will need to create a new variable
  equal to this difference.
\item
  Present a scatter plot of population weighted greenspace in 2021 over
  the greenspace in 2010.
\end{enumerate}

\textbf{BONUS OPPORTUNITY}: Use color-coding to differentiate urban
areas that added versus lost greenspace in that time. Then include a
45-degree line to further highlight the change.

\end{document}
